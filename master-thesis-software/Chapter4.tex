%Chapter 4

\renewcommand{\thechapter}{4}

\chapter{Summary}
This thesis research aims at whether texture features can be applied in SVM for classification of different types of diffraction images. A JAVA application is implemented for verifying two existing tools and eventually developed for calculating the GLCM texture features from diffraction image. Unlike two existing tools, the JAVA application is implemented with simply UI which allows users to modify the parameters - offset and angle direction without changing the code. We only consider the GLCM with offset 1 and the average over four angle directions, and other GLCM with different parameters can be examined in the future.\par  
Through this thesis research, we attained an SVM classifier with high accuracy of 80.33\% for dealing with the classification of diffraction images. In this research, a concept of diffraction image pair is introduced to better describe the representation of one type of diffraction image. In general, the common number of texture features of an image is 20. To increase the number, we added one more image captured at the same time but different direction with the polarizer. Practically, the two diffraction images still represent one type which is cell, debris or strip. By applying this method, the total feature parameters are increased to 40. Because of the data type of the diffraction image and invalid feature elimination, we finally have 32 texture features remained for the experiment. \par
To achieve the SVM classifier, we are required selecting the feature parameters it contains. There are two different methods for selecting feature parameters proposed. The first method is called forward propagation. By applying this method, the next feature is selected based on the accuracy variation of the classifier when adding the feature parameter. A feature combination is determined when the accuracy of the classifier reaches the peak. The primary disadvantage is that it may miss some feature parameters that are able to contribute to the classifier. It is proved by the result of applying all 32 feature parameters for SVM. Therefore, the second method called back propagation is proposed. In contrast to the forward propagation, the mechanism of the back propagation is that a feature is eliminated based on the accuracy variation of the classifier when removing the feature parameter. By applying this method, we achieve the SVM classifier with the highest accuracy of 65.67\% by using 28 texture features for classification.\par   
To better deal with the non-linear classification, we applied the RBF kernel. We examined all combination between parameter C and $\gamma$ and identified the parameter pair of ($2^{12}$, $2^{-3}$) to be the best choice. By applying this pair for the SVM classifier we selected, the accuracy of the classifier is 80.33\%. Through the cross-validation, the selected classifier is proved not having the overfitting issue. Therefore, we validated there are 28 texture features which can be used by SVM for classification.
