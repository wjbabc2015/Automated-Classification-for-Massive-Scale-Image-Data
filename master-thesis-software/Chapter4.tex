%Chapter 4

\renewcommand{\thechapter}{4}

\chapter{Summary}
This thesis research aims at automated classification of diffraction images based on their texture features. To achieve the goal, we wanted to develop and verify a tool for extracting texture feature from GLCM. We also wanted to validate whether the SVM can be applied for image classification by using texture features of diffraction images. In addition, we tried to use two diffraction images captured at the same time but from different directions to improve the configuration. Also, we wanted to utilize the SVM kernel method to improve the performance of the SVM classifier. This thesis research achieved all its objectives by performing the empirical study.\par
Through the empirical study, we performed a case study in selecting features by using the EFCS approach which was performed in Thati's research\cite{Thati}. Due to the results of the case study, we found the classification accuracy is 70.33\% by using six out of 17 feature parameters. By contrast, the classification accuracy is 80.33\% by using 28 out of 34 feature parameters in our experiment. The comparison indicates that using two diffraction images has higher performance in pattern recognition of diffraction image. \par
To extract texture features from GLCM, we developed an application in JAVA with simple but convenient UI. By performing functional testing for the newly developed application, we found bugs in component loading images. Through the debugging process, the JAVA application was verified. In contrast to two existing tools, the JAVA application provides a more convenient way to modify GLCM parameters. Furthermore, the JAVA application can combine two diffraction images which are related to each other into one data example. Finally, the JAVA application can process unlimited diffraction images at once. The difference in feature values between the JAVA application and the MATLAB tool is caused by the sequence in processing GLCM normalization. \par
 In SVM, we fully utilized the tool LIBSVM to select feature parameters with high accuracy. The first SVM classifier we achieved by using 11 out of 34 feature parameters. The classification accuracy is 61.67\%. However, when applying 32 out of 34 feature parameters in SVM, we obtained the accuracy of 64.17\%. This result indicates that there are more than 11 feature parameters that can be applied in SVM for classification. Therefore, based on the second classifier with an accuracy of 64.17\%, we tried to remove the negative feature parameters to improve the performance. Through the process, we obtained the third classifier with an accuracy of 65.67\% by using 28 out of 34 feature parameters. \par
Through studying the SVM kernel method, we examined various parameter pairs of (C, $\gamma$). By selecting the parameter pair of ($2^{12}$, $2^{-3}$), we obtain the improved classification  accuracy of 80.33\%. To analyze the confusion matrix, the selected classifier has high performance in selecting cell and debris type diffraction images with classification accuracy of 88.76\% and 88.75\%. However, by comparing with other existing research\cite{Rakesh}, the accuracy is not good. The primary reason for this situation is that the accuracy of the training data set used in this experiment cannot be guaranteed. The classifier has plenty of space for improvement by refining the training data set in the future. Nevertheless, the SVM has higher performance in classification than the neural network method because we compared our experiment results with other existing research using a neural network for classification\cite{Qayyum}. \par


