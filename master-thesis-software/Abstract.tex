%Abstract Page 

\hbox{\ }

\renewcommand{\baselinestretch}{2}
\small \normalsize
\begin{center}
\textbf{Automated Classification of Massive Scale Image Data}
\end{center}
\begin{center}
by
\end{center}
\begin{center}
Jiabin Wang
\end{center}
\begin{center}
December 2016
\end{center}
Director of Thesis:  Dr. Junhua Ding\\
Major Department:  Computer Science\par

\large \normalsize

The diffraction image is a useful method to facilitate the representation of tiny entities such as the cell. It provides an efficient way to analyze the 3D morphological features of biological cells.   However, the representation of diffraction images is so abstract that classifying them is challenging. When it comes to the massive amount of diffraction images, a manual classification for them can be time-consuming, and their accuracy cannot be guaranteed. This research focuses on the automated classification of diffraction images with high accuracy. In this research, gray level co-occurrence matrix (GLCM) that is a Statistical method for image texture analysis is used to extract texture features and the support vector machine (SVM) algorithm is applied for classification among three types of diffraction image based on image texture features. These types are cell, debris, and strip. \par
Two diffraction images which are captured at the same time but from different directions are combined together to improve the pattern recognition of the diffraction image. The diffraction image is processed by a developed JAVA application into a numerical data example which contains 34 texture features. The application is implemented with simple User Interface (UI) to facilitate user's operation of the application. In contrast to two existing tools implemented in MATLAB and C++, the JAVA application provides a new functionality that allows users to modify the primary parameters of GLCM without changing the code. A case study is performed for selecting feature parameters. From the case study, 28 out of 34 texture features are selected as feature parameters applied for the SVM. Thus, a stable SVM classifier is attained using these feature parameters. Finally, an improvement process is performed by identifying the parameter pair of Radial Basis Function (RBF) kernel. Through assigning the parameter pair with ($C = 2^{12}$,$\gamma = 2^{-3}$), the classification accuracy is improved to 80.33\%. As the confusion matrix shows, the SVM classifier we selected from the experiment has high performance in selecting the cell and debris image types. Their accuracy is 88.76\% and 88.75\%.   
\newpage
\mbox{}
