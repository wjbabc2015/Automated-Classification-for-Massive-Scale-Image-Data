%Abstract Page 

\hbox{\ }

\renewcommand{\baselinestretch}{2}
\small \normalsize
\begin{center}
\textbf{Automated Classification of Massive Scale Image Data}
\end{center}
\begin{center}
by
\end{center}
\begin{center}
Jiabin Wang
\end{center}
\begin{center}
December, 2016
\end{center}
Director of Thesis:  Dr. Junhua Ding\\
Major Department:  Computer Science\par

\large \normalsize

The diffraction image is a useful method to facilitate the representation of tiny entity such as the cell. However, it is so abstract that classifying them is challenged. When it comes to the massive amount of diffraction images, a manual classification for them can be time-consuming, as well as that the accuracy is unable to be guaranteed. This research focuses on the automated classification of diffraction images with high accuracy. In this research, gray level co-occurrence matrix (GLCM) that is a statistic method for image texture analysis is used to extract texture features and the support vector machine (SVM) algorithm is applied for selecting the feature parameters for classification among three types of diffraction image - cell, debris, and strip. In addition, a concept of diffraction image pair is introduced to describe an innovative representation of diffraction image types. \par
All diffraction image pairs are processed by a developed JAVA application into data examples which contain 34 texture features. The application is implemented with simple User Interface (UI) to facilitate user's operating. In contrast to two existing tools implemented in MATLAB and C++, the JAVA application provides a new functionality that allows users to modify the primary parameters of GLCM without changing the code. Furthermore, a new approach for selecting feature parameters is proposed. By applying the approach, 28 out of 34 texture features are selected as feature parameters for SVM. Thus, a classifier with accuracy 65.67\% is attained using these feature parameters. Finally, a process for identifying the parameter pair of Radial Basis Function (RBF) kernel is performed. By assigning the parameter pair with ($C = 2^{12}$,$\gamma = 2^{-3}$), the accuracy is improved to a higher accuracy of 80.33\%. As the confusion matrix shown, the SVM classifier we selected from the experiment has high performance in selecting the cell and debris image types. The accuracy of them is 88.76\% and 88.75\%.   
\newpage
\thispagestyle{empty}
\mbox{}
