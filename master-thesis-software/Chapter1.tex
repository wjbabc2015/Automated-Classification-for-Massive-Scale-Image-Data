%Chapter 1

\renewcommand{\thechapter}{1}

\chapter{Introduction}

Big data is a term that describes the large amount or high complexity of data sets. With the rapid development of information technology, the concept of big data becomes more and more popular. Since the data sets are large or complex, it is inadequate for traditional data processing application to handle them\cite{Snijders}. Image data is a direct representation of entity or object using pictorial information and widely applied in many research fields because of that, such as biology, physics, geology, medicine and so on. The diffraction image is a kind of image data that makes the tiny entity more visible to the observer. However, it is so abstract that it is challenged for observers to classify them. When it comes to the concept of big data, manual processing them is time-consuming, and the accuracy which depends on the experience and skills of observers is unable to be guaranteed. Based on the concept of digital imaging, an image data can be represented by physical points called pixel\cite{Foley}. With the resolution of image enhanced, the analysis for image becomes tougher and tougher. Therefore, analyst involves the texture analysis which is a common method that is widely used to analyze the image texture that can provide the spatial relationship of intensities in an image or selected segment of an image\cite{Linda}. Within all the highest methods of texture analysis, gray level co-occurrence matrix (GLCM) is the most useful approach because it is easy for the analyst to learn and extensively applied\cite{Nanni}. To perform the method, GLCM is created by counting the spatial relation between two pixels over the whole image, and the texture of GLCM, such as contrast, correlation, energy, homogeneity, are calculated from the GLCM. To validate whether or not the texture features can be used for image classification, a method called machine learning is introduced. Machine learning is one of the state-of-the-art approaches to deal with analyzing the massive amount of data. Practically speaking, machine learning is the process that provides the computer the ability to grow and improve by itself with certain data sets\cite{Mitchell}. It constructs an algorithm that is enabled to learn from and predict data\cite{Mehryar}. By applying the concept of machine learning, the computer is enabled to process a massive amount of specific data with the corresponding model. However, to achieve the model used in machine learning, it requires certain data to train. Support vector machine (SVM) algorithm is a kind of supervised learning model in machine learning to facilitate classification, regression, or other tasks\cite{Cortes}. In SVM, given a set of labeled data, the SVM algorithm can build a classifier to predict new unlabeled data. In general, the training data is a vector contains multiple feature parameters. For instance, a tumor diagnosis model is trained on a certain amount of data contains patient's age and tumor's size, as well as the corresponding label which is either benign or malignant. By providing the unknown data with the same format, the model is able to label the example either benign or malignant. Thus, if we apply this approach in analyzing image data, we can train the SVM classifier using GLCM texture features for classification. \par 
In this thesis research, we apply GLCM for image texture analysis and SVM algorithm for the image classification. GLCM is a useful approach to representing an image by considering the spatial relationship between two pixels of the image in each time. In this thesis research, we estimate 17 texture feature parameters\cite{Haralick} for each image, and gather two images captured at the same time but from different directions using the polarizer to represent a type of diffraction image. Therefore, 34 texture features are extracted for SVM. We have two tools which have been developed for calculating texture features in MATLAB and C++, but they provide different feature results from the same image input. Thus, a JAVA application  with a simple interface is developed in the research to verify them. In addition, through a series of verification processes, the new JAVA application is verified, instead of the MATLAB script and C++ tool. By using this JAVA application, a set of examples are calculated from provided diffraction image data for image classification procedure. By applying SVM, we select a classifier with high accuracy for classification among three diffraction image types - cell, debris, and strip. To achieve the classifier, we propose two approaches with the different mechanism. The first approach is called forward propagation. By using this approach, the accuracy increases by adding new feature parameter. We can achieve the classifier when the accuracy on longer increases by adding more feature parameters. The limitation of the forward propagation approach is proved by the second approach called backward propagation. By using the approach, a classifier is selected by removing feature parameters. The classifier is selected using 28 feature parameters for image classification with accuracy 65.67\%. To optimize the parameter pair of RBF kernel to ($2^{12}$,$2^{-3}$), the accuracy is improved to 80.33\%. We also perform the 10-folder cross-validation for the selected classifier, and the result presents the selected classifier is stable and able to generalize to new examples.\par
In this thesis, a new JAVA application is developed. In contrast to the tools implemented in MATLAB and C++, the JAVA application provides a new functionality for users to modify the GLCM parameters - offset and angle direction - without changing the code. By using this application, the GLCM can be created in any offset along any directions. Unlike the previous research\cite{Thati}, this thesis provides the statistic approach to select feature parameters for SVM classifier. In addition, the thesis research adds one more diffraction image captured from another direction to enhance the performance of the selected SVM classifier. \par
This thesis is organized into the following chapters. Chapter 2 mainly presents the GLCM. In this chapter, we present a detailed description of GLCM is provided. Following the description, the thesis demonstrates the development of GLCM texture feature calculation application in JAVA and a procedure to verify all the applications. In the end of chapter 2, we make a brief conclusion. Chapter 3 provides the procedure for selecting SVM classifier. Within this chapter, it starts with experiment data preprocessing. Then, it describes the experiment method we proposed step by step with the result of each step. It also discusses the validation of the SVM classifier, and we propose another method for selecting the SVM classifier to mitigate the limitation of the first approach. This chapter also presents the SVM kernel and the parameter pair of RBF kernel, and we demonstrate the modification process to achieve the highest accuracy of the classifier. Finally, in chapter 4, the thesis provides an overall summary of the whole thesis.      