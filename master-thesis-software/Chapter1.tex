%Chapter 1

\renewcommand{\thechapter}{1}

\chapter{Introduction}

Big data is a term that describes the large amount or high complexity of data sets. With the rapid development of information technology, the concept of big data becomes more and more prevalent. Since the data sets are large or complex, traditional data processing applications are inadequate to handle them\cite{Snijders}. Image data is a direct representation of an entity or object using pictorial information and, because of that, is widely applied in many research fields such as biology, physics, geology, medicine and so on. The diffraction image is a kind of image data that makes tiny entities more visible to the observer. For instance, it provides an efficient way for analysis of three-dimensional (3D) morphological features of biological cells\cite{Brown}. However, the representation of diffraction images is so abstract that it is challenging for observers to classify them. When it comes to the concept of big data, manually processing them is time-consuming, and the accuracy which depends on the experience and skills of observers cannot be guaranteed. Based on the concept of digital imaging, image data can be represented by physical points called pixel\cite{Foley}. With the resolution of images enhanced, the analysis for images becomes tougher and tougher. Therefore, analyst involves texture analysis which is a common method that is widely used to analyze image texture that can provide the spatial relationship of intensities in an image or selected segment of an image\cite{Linda}. Within all the best methods of texture analysis, gray level co-occurrence matrix (GLCM) is the most useful approach because it is easy for the analyst to learn and be extensively applied\cite{Nanni}. To perform the method, GLCM is created by counting the spatial relation between two pixels over the whole image, and the texture of GLCM, such as contrast, correlation, energy, or homogeneity, is calculated from the GLCM. Machine learning is one of the state-of-the-art approaches to deal with analyzing the massive amount of data. Practically speaking, machine learning is the process that provides the computer the ability to grow and improve by itself with certain data sets\cite{Mitchell}. It constructs an algorithm that enables the computer to learn from and predict data\cite{Mehryar}. By applying the concept of machine learning, the computer is enabled to process a massive amount of specific data with the corresponding model. However, to achieve the model used in machine learning requires certain data to train. Support vector machine (SVM) algorithm is a kind of supervised learning model in machine learning to facilitate classification, regression, or other tasks\cite{Cortes}. In SVM algorithm, given a set of labeled data, the SVM algorithm can build a classifier to predict new unlabeled data. In general, the training data in a vector contains multiple feature parameters. For instance, a tumor diagnosis model is trained on a certain amount of data that contains patient's age and the tumor's size, as well as the corresponding label which is either benign or malignant. By providing the unknown data with the same format, the model is able to label the example either benign or malignant. To validate whether the SVM algorithm can be applied for image classification based on texture features extracted from GLCM, a case study was conducted. Finally, kernel methods are widely applied in SVM, which enables SVM to be competitive with neural networks on some tasks such as handwriting recognition\cite{Aizerman}.   \par 
This thesis research focuses on applying the SVM algorithm for classification based on texture features of diffraction images. First, we want to develop and verify a GLCM calculation tool for extracting texture features because the existing tools cannot be satisfied with the requirements. Then, we want to validate whether the SVM algorithm can be applied for classification by using texture features of the diffraction image. Finally, we want to validate whether the parameters of kernel methods affect the performance of the SVM classifier. \par
Through this thesis study, we have developed a GLCM application for extracting texture features from diffraction images. In contrast to two existing tools implemented in MATLAB and C++, the newly developed application provides simple User Interface (UI) for users. Thus, users who do not have programming skills can modify the parameter offset and parameter angle direction without changing the code when creating the GLCM. Furthermore, we have verified the newly developed application through functionality testing. Based on the results of testing, we have debugged some defects in the component which is used to load the image data and convert image data to a numerical pixel matrix. We also have verified that the difference between tools implemented in MATLAB and JAVA is caused by the way the applications realize the texture features from GLCM. In the JAVA application, we implemented the common algorithms which are defined by Haralick for calculating texture features\cite{Haralick}. We also have validated that the SVM can be applied for classification based on the texture feature of the diffraction image. Instead of using an extensive feature correlation study and fully automated algorithm CFS for selecting feature parameters, the validation process has fully utilized the tool LIBSVM to select feature parameters and examine how each texture feature or texture feature combination affects the classification accuracy of SVM. In addition, we have selected the SVM classifier with the highest accuracy by modifying the parameters of radial basis function (RBF) kernel. Finally, we have proved that the representation of diffraction images can be improved by using two diffraction images which are captured at the same time but from different directions. \par
This thesis is organized into the following chapters. Chapter 2 mainly presents the GLCM. In this chapter, we present a detailed description of GLCM. Following the description, the thesis demonstrates the development of GLCM texture feature calculation application in JAVA and a procedure to verify all the applications. In the end of chapter 2, we draw a brief conclusion. Chapter 3 provides the procedure for selecting an SVM classifier. This chapter, it starts with experiment data preprocessing. Then, it describes the experiment method we proposed step by step with the results of each step. It also discusses the validation of the SVM classifier, and we propose another method for selecting the SVM classifier to mitigate the limitations of the first approach. This chapter also presents the SVM kernel and the parameter pair of the RBF kernel, and we demonstrate the modification process to achieve the highest accuracy of the classifier. Finally, chapter 4 provides a summary of the whole thesis.      